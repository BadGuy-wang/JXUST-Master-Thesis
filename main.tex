% ============================
%|                            |
%|         导入各种包			  |
%|                            |
% ============================
\documentclass[hyperref,UTF8,a4paper]{ctexart}
\usepackage{pdfpages} 
\usepackage{amsmath,amssymb} 
\usepackage{graphicx}
\usepackage{xeCJK}
\usepackage{xeCJKfntef}
\usepackage{setspace}
\usepackage{bm}
\usepackage{booktabs}                 % 做三线表的包
\usepackage{multirow} 
\usepackage{tikz}
\usepackage{longtable}
\usepackage{booktabs}
\usepackage{tabularx, makecell}
% 设置页边距
\usepackage{geometry} 
\geometry{left=2.5cm,right=2.5cm,top=3.5cm,bottom=2.5cm}

% 设置页眉的包
\usepackage{fancyhdr} 
\pagestyle{fancy}	
\fancyhf{}
% 设置页码
\cfoot{\thepage}	

% 取消图片注释中的冒号
\usepackage{caption}
\DeclareCaptionLabelSeparator{twospace}{\ ~} 
\captionsetup{labelsep=twospace}

% 将章节序号改为中文
\usepackage{zhnumber} 
\renewcommand \thesection {第\zhnum{section}章}
\renewcommand \thesubsection {\arabic{section}.\arabic{subsection}}

%%%============================字体设置========================%%%
\setmainfont{Times New Roman}        %缺省英文字体 Times New Roman
\setCJKmainfont[Path=Configuration/]{simsun.ttf}          %宋体
\setCJKsansfont[Path=Configuration/]{simhei.ttf}          %黑体
\setCJKmonofont[Path=Configuration/]{simsun.ttf}          %宋体
%------------------------------------------------------------------------
\setCJKfamilyfont{song}[Path=Configuration/]{simsun.ttf}  %导入宋体字体
\newcommand{\song}{\CJKfamily{song}} %设置宋体快捷命令
%------------------------------------------------------------------------
\setCJKfamilyfont{kai}[Path=Configuration/]{simkai.ttf}   %导入楷书字体
\newcommand{\kai}{\CJKfamily{kai}}   %设置楷书快捷命令
%------------------------------------------------------------------------
\setCJKfamilyfont{hei}[Path=Configuration/]{simhei.ttf}   %导入黑体字体
\newcommand{\hei}{\CJKfamily{hei}}   %设置黑体快捷命令

%%%============================摘要及关键词========================%%%
\newcommand{\cnabstractname}{\hei \zihao{-3} \textbf{摘要}}
\newcommand{\enabstractname}{\zihao{-3} \textbf{Abstract}}
\newcommand\cnkeywordsname{关键词}
\newcommand\cnkeywords[1]{ {\noindent\hei\zihao{-4}\cnkeywordsname: }#1}
\newcommand\enkeywordsname{Key Words}
\newcommand\enkeywords[1]{ {\noindent\bfseries\zihao{-4}\enkeywordsname: }#1}

\newenvironment{cnabstract}{%
 \newpage
 \mbox{}
 \par 
 \noindent\mbox{}\hfill{\bfseries \ziju{2}{\cnabstractname}}\hfill\mbox{}\par 
 \vspace{0.64cm}
 }
    
\newenvironment{enabstract}{%
 \newpage
 \mbox{}
 \par 
\noindent\mbox{}\hfill{\bfseries \ziju{2}{\enabstractname}}\hfill\mbox{}\par
  \vspace{0.64cm}
 }

%%%============================设置目录内容格式============================%%%
\usepackage{titletoc}
\renewcommand{\contentsname}{\hei \zihao{3} \textbf{目\quad 录} \vspace{2ex}}
\titlecontents{section}[4em]
    {
        \bfseries %字体加粗
        \hei
        \zihao{-4} %小四号字体
        \vspace{6pt} %调整上下距离
    } 
    {\contentslabel{3em}} %调整
    {\hspace*{-3em}} %调整左右距离
    {~\titlerule*[0.3pc]{$.$}~\contentspage} %调整连接点粗细
\titlecontents{subsection}[5em]
    {	
    	\bfseries %字体加粗
    	\hei
        \zihao{-4} 
        \vspace{6pt}
    } 
    {\contentslabel{2em}} 
    {\hspace*{-2em}}
    {~\titlerule*[0.3pc]{$.$}~\contentspage}
\titlecontents{subsubsection}[7.5em]
    {
        \zihao{-4} 
        \vspace{6pt}
    } 
    {\contentslabel{2.5em}} 
    {\hspace*{-2em}}
    {~\titlerule*[0.3pc]{$.$}~\contentspage}
%%%============================设置正文各级标题格式============================%%%
\CTEXsetup[format={\centering \heiti \zihao{3} \bfseries},beforeskip={24bp},afterskip={18bp},aftername={\enspace}]{section} % 一级标题三号黑体加粗
\CTEXsetup[format={\raggedright \heiti  \zihao{4} \bfseries},beforeskip={24bp},afterskip={6bp},aftername={\enspace}]{subsection} % 二级标题四号黑体加粗
\CTEXsetup[format={\raggedright \heiti  \zihao{4}},beforeskip={12bp},afterskip={6bp},aftername={\enspace}]{subsubsection} % 三级标题四号黑体
%%%============================设置参考文献===================================%%% 
% 设置参考文献的包
\usepackage{gbt7714} % 使用GB/T-7714-2015格式 
% 设置参考文献行间距
\usepackage{natbib}
\setlength{\bibsep}{0.5ex}

% ========================
%|                        |
%|         正文			  |
%|                        |
% ========================

\begin{document}
%%%=============================加入封面和诚信承诺书=============================%%%
% 导入封面
%pdf修改为论文封面的PDF文件绝对路径名
\includepdf[pages=-]{江西理工大学硕士论文封面.pdf}
%%%===============================加入中英文摘要.===============================%%%
%摘要开始部分
\setcounter{page}{0}						% 设置当前页页码编号从1开始计数
\pagenumbering{Roman}						% 设置页码字体为小写阿拉伯字体
\fancyhead[C]{江西理工大学硕士学位论文\quad 摘要} % 设置学院
\renewcommand{\headrulewidth}{0.5pt} %设置页眉横线粗细
\addcontentsline{toc}{section}{摘要}\tolerance=500 %将摘要放进目录
\section*{\centering \hei \zihao{-3} 摘\quad 要}
\song \zihao{-4} \setstretch{1.523} %1.523是22pt
本模板为成都理工大学地球物理学院本科毕业论文~\LaTeX~模板。

本文主要介绍了模板的各文件含义及使用方法,并简略介绍了如何使用~\LaTeX~进行论文写作,指明了使用本模板进行本科生毕业论文写作的基本方法,另外还强调了图片表格插入的一些细节问题, 希望这些对使用者能有所帮助。

\par 
\mbox{}
\par 
\hei \textbf{关键字:} \song 关键字1,关键字2,关键字3

\newpage
\clearpage
\addcontentsline{toc}{section}{Abstract}\tolerance=500 %将摘要放进目录 
\section*{\centering \zihao{3} Abstract}
\zihao{-4} \setstretch{1.384} 
This template is the ~\LaTeX~ template for the graduation thesis of undergraduates from the School of Geophysics, Chengdu University of Technology.

This article mainly introduces the meaning of each file of the template and how to use it, and briefly introduces how to use ~\LaTeX~ to write a thesis. It points out the basic method of using this template for undergraduate graduation thesis writing, and also emphasizes the picture form. Some details of the insertion, I hope these can be helpful to the user.

\par
\mbox{}
\par 
\textbf{Keywords:} keyword1, keyword2, keyword3
%%%===============================加入目录=====================================%%%
\clearpage
\fancyhead[C]{江西理工大学硕士学位论文\quad 目录} % 设置学院
\renewcommand{\headrulewidth}{0.5pt} %设置页眉横线粗细
\setcounter{page}{1}						% 设置当前页页码编号从1开始计数
\pagenumbering{Roman}						% 设置页码字体为大写罗马字体
\tableofcontents
%%%===============================加入正文=====================================%%%
% 设置正文页眉&页码
\newpage
\fancyhead[C]{\song \zihao{5} \leftmark} 					% 设置章节标题
\renewcommand{\headrulewidth}{0.5pt} 		% 设置页眉横线粗细、
\setcounter{page}{0}						% 设置当前页页码编号从1开始计数
\pagenumbering{arabic}						% 设置页码字体为小写阿拉伯字体
\song \zihao{-4}

\section{模板使用指南}

\subsection{编译工具的选择}

\subsubsection{Overleaf线上编译平台}

鉴于~\LaTeX~环境搭建的繁琐性,推荐在Overleaf这一线上平台使用本模板。

进入\href{https://cn.overleaf.com/}{https://cn.overleaf.com/}{Overleaf}之后需使用邮箱注册一个账号来使用免费版的Overleaf。

注册完成之后进入项目页面,点击左侧\textbf{<创建新项目>}按钮,选择最后一项\textbf{<预览所有>},在跳转后的页面内的搜索框内输入“成都理工大学本科论文模板”。

点击第一个结果项,进入模板详情页后点击\textbf{<Open as Template>}按钮,即可打开本模板。

由于Overleaf模板审查的存在延迟,不能保证Overleaf里的模板是最新的,所以在这里也附上\href{https://github.com/fumeng6/CDUT-Undergraduate-Thesis-Template}{https://github.com/fumeng6/CDUT-Undergraduate-Thesis-Template}{github上本模板的链接}。在github上下载好本模板的压缩包后,使用Overleaf创建新项目中的上传项目选项即可在线编译本模板。

\subsubsection{本地环境搭建及编译器}

如想在本地进行模板使用编辑,可前往\href{https://tug.org/texlive/}{https://tug.org/texlive/}{TeX Live 的官方站点}下载发行版本的TeX Live以进行本地的编译工作,本模板写作所用版本为2020版,故推荐下载TeX Live 2020。

现今发行版本的TeX Live自带的Texworks编辑器可能过于老旧,读者可以可参照《\href{https://www.jianshu.com/p/3e842d67ada2}{https://www.jianshu.com/p/3e842d67ada2}{latex零基础入门}》这篇文章进行~\LaTeX~的编译环境的准备以及编译器的配置。

\subsection{模板各文件说明}

\begin{description}
  \item[figures]  此文件夹内存放论文中所需插入的图片,后续需要读者自行添加.
  
  \item[configuration]  此文件夹内存放论文中所用的文档类、字体以及诚信承诺书等,不可修改.
  
  \item[CDUT Bachelor thesis.tex]  此文件为模板主文件,在完成撰写后,编译它将生成读者的论文.
  
  \item[chapter1.tex]  此为论文第一章文件,后续需要读者自行修改其中内容.
  
  \item[chapter2.tex]  此为论文第二章文件,后续需要读者自行修改其中内容.
  
  \item[chapter3.tex]  此为论文第三章文件,后续需要读者自行修改其中内容.
  
  \item[abstract.tex]  此为中、英文摘要文件,后续需要读者自行修改其中内容.
  
  \item[conclusion.tex]  此为论文的结论文件,后续需要读者自行修改其中内容.
  
  \item[thanks.tex]  此为论文的致谢文件,后续需要读者自行修改其中内容.
  
  \item[ref.bib]  此文件为模板文献数据库文件,其中储存论文中所引用的参考文献,后续需读者自行修改添加.
\end{description}

\subsection{文档具体使用步骤}

\begin{description}
  \item[Step 1]  在Overleaf上打开模板后,请点击页面左上角菜单,下拉到设置板块,将编译器设为XeLaTex,Tex Live版本选择2020.
  
  \item[Step 2]  abstract.tex、conclusion.tex、thanks.tex这三个文档,分别对应着中文及英文摘要、结论及致谢三个部分的论文内容,请读者自行撰写.

  \item[Step 3]  打开主文档CDUT Bachelor thesis.tex, 于“封面页信息采集”板块填写题目、作者者等封面页信息.

  \item[Step 4]  chapter1.tex、chapter2.tex、chapter3.tex这三个文件是笔者预先编写,它们依次对应着本文的第一章、第二章以及第三章,读者需要自行修改撰写其中的内容,这也即是论文的正文部分写作(这一步涉及到的诸如公式、表格、图片的插入,参考文献的引用等问题将在后面做详细说明),如需进行更多章节的写作请照例创建新的chapter4、chapter5等文件,并在主文档中照例引用即可.
  
  \item[Step 5]  在全部完成之后,再进行最后一次编译,确认无误后点击PDF预览板块上方的下载按钮,即可将写好的论文下载到本地.
\end{description}

\subsection{Overleaf的简单使用指南}

总的来说Overleaf的使用并不复杂,如果看不懂英文界面,一般打开项目页的时候页面顶部会有一个蓝色对话框,点击确认则整个Overleaf即进入中文版面。

\subsubsection{Overleaf上的常用快捷键}

首先介绍一下Overleaf上的常用快捷键。

\begin{table}[ht]\centering
\begin{tabular}{l l l l}
\hline
Ctrl + F & 查找(并替换)& Ctrl + Enter & 编译 \\
Ctrl + Z & 撤销 & Ctrl + Y & 恢复撤销 \\
Ctrl + Home & 跳转到文件开头 & Ctrl + End & 跳转到文件末尾 \\
Ctrl + L & 转到某行 & Ctrl + D & 删除当前行 \\
Ctrl + U & 改为大写 & Ctrl + Shift + U & 改为小写 \\
Ctrl + B & 粗体 & Ctrl + I & 斜体 \\
\hline
\end{tabular}
\end{table}
特别说明,在LaTeX中加粗某些字使用的是\verb|\textbf{}|命令,这里的加粗快捷键的作用就是生成这一命令,但是在诸如公式等数学环境下需要加粗某字母时,应使用的命令是\verb|\mathbf{}|。

\subsubsection{历史记录}

Overleaf具有历史记录功能,它在编译页面的右上角,里面可以看到之前版本的代码,这在改错代码或者想要找之前某版内容的时候这是一个极有用的功能。

\subsubsection{上传与新建文件}

在使用模板的过程中可能会需要用到的上传图片功能,该按钮在编译页面的左上角,与之并排的还有新建文件按钮(若正文超过三章,则需新建)以及新建文件夹按钮。

或者,读者也可以在对应的文件夹上右键鼠标,选择上传或新建文件。

\subsubsection{双向定位}

在\textbf{<重新编译>}按钮的左下方有上下两个排列在一起,分别指向左侧代码工作区以及右侧PDF预览区的两个箭头,当读者在代码工作区选中一行内容并点击指向右侧的箭头,Overleaf会将PDF页面跳转到这一行代码区内容所指向的PDF区内容,简单来说就是由代码跳转到编译结果。反过来,由PDF区跳转到代码区也是一样的操作。或者读者可以简单的在想要跳转的内容上双击鼠标左键(这一操作仅在由PDF内容跳转到代码内容时有效)。

\subsubsection{更改PDF阅读器}

对于用Overleaf内置的PDF阅读器感到不舒服的读者,可以在菜单中将阅读器由<\textbf{内嵌}>改为<\textbf{本机}>,这样一来编译之后的PDF预览应当用的是读者的浏览器所带有的PDF阅读器。笔者所使用的是Google Chrome,其自带的PDF阅读器相较Overleaf的内嵌PDF阅读器在功能上会强大很多。不过需要注意的是,使用本机阅读器之后就无法再使用Overleaf的双向定位功能。

 % 导入第一章内容

\section{\LaTeX 使用入门}

~\LaTeX~的命令形式均为一个反斜杠后跟字母及括号,一般请在输入一个命令之后敲一个空格,以免产生未知错误。

\subsection{换段、换行与空格}

在~\LaTeX~中,单个的回车与空格会被忽略。

\begin{description}
 \item[换段]此处的换段指的是开启一个新段落,通过空一行(两次回车)实现段落换行,也可以通过 \verb|\par| 命令来新起一段。
 \item[换行]此处的换行指的是换行不换段,通过输入两个反斜杠也即是\verb|\\|可以实现换行,换行之后不会缩进,与上一行仍属于一个段落的内容。
 \item[空格]此处的空格指的不是键盘上敲一下空格键生成的空格,而是出现在文中的空格,例如\quad 这\quad 样,通过命令\verb|\quad|即可实现。
\end{description}

\subsection{标题}

\begin{tabular}{l l}
\verb|\chapter{}| & 一级标题命令\quad 形如第一章 \\
\verb|\section{}| & 二级标题命令\quad 形如1.1 \\
\verb|\subsection{}| & 三级标题命令\quad 形如1.1.1 \\
\verb|\subsubsection{}| & 四级标题命令\quad 形如1.1.1.1 \\
\end{tabular}

\subsection{字体调节}

\begin{tabular}{l l l}
 \verb|\textbf{}| & \textbf{成都理工大学} & 文本加粗\\
 \verb|\textit{}| & \textit{成都理工CDUT} & 文本斜体\\
 \verb|\mathbf{}| & $\mathbf{CDUT}$ & 数学环境下加粗\\
 \verb|\bm{}| & $\bm{CDUT}$ & 数学环境下加粗斜体\\
\end{tabular}

\subsection{公式}

\subsubsection{常用数学环境}

在~\LaTeX~中撰写公式,需要给公式加上一个数学环境,下列为常用的数学环境。

\begin{tabular}{ll}
\toprule
环境命令 & 命令含义\quad 是否编号\\
\midrule
 \verb|\(...\)| & 行内公式\quad 不编号 \\
 \verb|$...$| & 行内公式\quad 不编号 \\
 \verb|\begin{math}...\end{math}| & 行内公式\quad 不编号 \\
 \verb|\[...\]| & 行间公式\quad 不编号 \\
 \verb|\begin{equation}...\end{equation}| & 行间公式\quad 编号 \\
 \verb|\begin{displaymath}...\end{displaymath}| & 行间公式\quad 不编号 \\
 \verb|\begin{equation*}...\end{equation*}| & 行间公式\quad 不编号 \\
\bottomrule
\end{tabular}

\subsubsection{公式范例}

\subsubsection{行内及行间公式}

在文中引用公式可以这么写:$a^2+b^2=c^2$这是勾股定理,它还可以表示为$c=\sqrt{a^2+b^2}$,还可以让公式单独一段并且加上编号。注意,公式前请不要空行。
\begin{equation}
\sin^2{\theta}+\cos^2{\theta}=1 \label{eq:pingfanghe}
\end{equation}

这样就可以通过添加标签在正文中引用公式,如式\eqref{eq:pingfanghe}。

\subsubsection{矩阵及多行对齐公式}

在大的数学环境内嵌套一些特定环境可以实现更加丰富的效果,下面的示例都是在\verb|\begin{equation}...\end{equation}|中嵌套。

嵌套\verb|\begin{matrix}...\end{matrix}|可实现矩阵的显示:
\begin{equation}
  \mathbf{A}=
  \left[\begin{matrix}
    1&2&3&4\\
    11&22&33&44\\
  \end{matrix}\right] \times
  \left[\begin{matrix}
    22&24\\
    32&34\\
    42&44\\
    52&54\\
  \end{matrix}\right]
\end{equation}

嵌套\verb|\begin{aligned}...\end{aligned}|可实现多行对齐的公式:
\begin{equation}
  \begin{aligned}
    f_1(x)&=(x+y)^2\\
          &=x^2+2xy+y^2
  \end{aligned}
\end{equation}

上面的形式同样也能通过嵌套\verb|\begin{split}...\end{split}|实现:
\begin{equation}
\begin{split}
A & = \frac{\pi r^2}{2} \\
 & = \frac{1}{2} \pi r^2
\end{split}
\end{equation}

在公式的对齐中,使用的都是\verb|&|字符来标定对齐位置,~\LaTeX~会将\verb|&|后的第一个字符内容对齐,在两行之间使用\verb|//|来分割。

\subsubsection{对齐方程组}

方程的对齐可以用\verb|\begin{align*}...\end{align*}|实现,这一环境作用为生成一不带编号(带*号表示不编号,去掉*号则对每行公式都编号,此规则适用大多数的数学环境)的多行对齐的方程。
\begin{align*} 
2x - 5y &=  8 \\ 
3x + 9y &=  -12
\end{align*}

下面是一个灵活插入\verb|&|以实现更复杂的方程组排列对齐的例子:
\begin{align*}
x&=y           &  w &=z              &  a&=b+c\\
2x&=-y         &  3w&=\frac{1}{2}z   &  a&=b\\
-4 + 5x&=2+y   &  w+2&=-1+w          &  ab&=cb
\end{align*}

\subsubsection{长公式}

当一个公式过长时可以通过\verb|\begin{multline*}...\end{multline*}|将其分割为两行显示,带*号表示此环境不对公式进行编号。

\begin{multline*}
p(x) = 3x^6 + 14x^5y + 590x^4y^2 + 19x^3y^3- 12x^5y^7 - 12y^3 + 2y^4 - a^2b^5\\ 
- 12x^2y^4 - 12xy^5 + 2y^6 - a^3b^3
\end{multline*}

\subsection{图片}

\subsubsection{\LaTeX 中的图}

\LaTeX 环境下可以使用常见的图片格式:JPEG、PNG、PDF、EPS等。当然也可以使用\LaTeX 直接绘制矢量图形,可以参考pgf/tikz等包中的相关内容。需要注意的是,无论采用什么方式绘制图形,首先考虑的是图片的清晰程度以及图片的可理解性,过于不清晰的图片将可能会浪费很多时间。

\subsubsection{插入图片}

插入图片的命令为\verb|\includegraphics[]{}|,方括号[]里是控制图片大小以及角度的控制命令,大括号{}里是所插入图片的文件名(注意,一定要把图片放到figures文件夹里去)。
\subsubsection{插图控制}

这里罗列一些控制图片的常见命令:

\begin{tabular}{ll}
\verb|scale=...| & 等比例缩放 \\
\verb|width=..., height=...| & 宽和高 \\
\verb|width=...\textwidth| & 与文本宽度成比例\\
\verb|angle=...| & 旋转(顺时针为负数,逆时针为正数,单位度)\\
\end{tabular}

\subsubsection{图片环境}

在\LaTeX 中插入的图片需要放入一个名为浮动体的环境中以便于\LaTeX 进行排版,其一般形式如下:

\verb|\begin{figure}[!htbp]|\par
\verb|\centering|\par
\verb|\includegraphics[]{}|\par
\verb|\chartname{图名}|\par
\verb|\label{图片标签}|\par
\verb|\end{figure}|

简单来说,\verb|\begin{figure}...\end{figure}|是一对,就像C语言里if和end一样,在if和end之间称循环体,在\verb|\begin{figure}...\end{figure}|之间为图片浮动体。

\verb|\chartname{图名}|这一命令即是对插入的图片取一个图名。

\verb|\label{图片标签}|这个命令紧跟在图名命令后,作用是给图片加一个标签,所谓标签也即是引用的时候你输入这个标签就能够引用图名了。

\verb|[htbp]|选项意即是浮动体位于此处、页顶、页底、独立一页。

\subsubsection{插图范例}

下面做一些简单的示例:

\begin{figure}[!htb]
  \begin{minipage}[t]{0.5\linewidth}
    \centering
     \includegraphics[width=0.4\textwidth]{figures/images1.jpg}
     \caption{我书读得多,不会骗你}\label{fig:1}
  \end{minipage}%
  \begin{minipage}[t]{0.5\linewidth}
    \centering
    \includegraphics[scale=0.5,angle=-45]{figures/images2.jpg}
    \caption{哈哈哈}\label{fig:2}
  \end{minipage}
\end{figure}

\subsubsection{\LaTeX 常用的长度单位及度量}

考虑到插入图片需求的多样性,下面罗列了一些可以用于前面图片控制命令的\LaTeX 长度单位及命令,在具体使用的时候可以多尝试一下,以达到最佳效果。

\begin{tabular}{ll}
pt & 单位长度,大约为0.3515mm \\
mm & 一毫米 \\
cm & 一厘米 \\
in & 一英寸 \\
ex & 当前字体尺寸中x的高度 \\
em & 当前字体尺寸中M的宽度 \\
\verb|\columnsep| & 两列之间的距离 \\
\verb|\columnwidth| & 列宽 \\
\verb|\linewidth| & 当前环境中线的宽度 \\
\verb|\paperwidth| & 页面宽度 \\
\verb|\paperheight| & 夜面高度 \\
\verb|\textwidth| & 文本宽度 \\
\verb|\textheight| & 文本高度 \\
\end{tabular}

\subsection{表格}

\subsubsection{通过在线工具生成表格}

表格的输入可能会比较麻烦,可以使用在线的工具,如~\href{https://www.tablesgenerator.com/}{https://www.tablesgenerator.com/}{Tables Generator}~能便捷的创建表格,也可以使用离线的工具,如~\href{https://ctan.org/pkg/excel2latex}{https://ctan.org/pkg/excel2latex}{Excel2LaTeX}~支持从Excel表格转换成\LaTeX{}表格。\href{https://en.wikibooks.org/wiki/LaTeX/Tables}{https://en.wikibooks.org/wiki/LaTeX/Tables}{LaTeX/Tables}~上及~\href{https://www.tug.org/pracjourn/2007-1/mori/mori.pdf}{https://www.tug.org/pracjourn/2007-1/mori/mori.pdf}{Tables in LaTeX}~也有更多的示例能够参考。

\subsubsection{表格的一般形式}

表格环境一般格式如下:

\verb|\begin{table}[htbp]|\par 
\verb|\centering|\par 
\verb|\chartname{表名}|\par 
\verb|\label{表格标签}|\par 
\verb|\begin{tabular}{c c}|\par 
\verb|...&...\\|\par 
\verb|...&...|\par 
\verb|\end{tabular}|\par 
\verb|\end{table}|\par 

首先是关于插入表格的位置控制问题。

同前面的图片的插入一样,主要就是用$h,t,b,p$这四个选项来控制表格位于此处、页顶、页底及单独一页,笔者注意到从Tables Generator粘贴来的代码是没有给这一浮动体位置控制命令的,因此需要读者自己添加。

因为地物院的论文写作要求中图名位于下方,表名位于上方,所以在写表格的时候应当将生成表格标题以及表格标签的命令放在表格内容开始之前,也即是上面一般形式中的位置。

\subsubsection{分割单元格及单元格内容对齐}

表格的主体内容在\verb|\begin{tabular}{c l r}...\end{tabular}|之间,在两列之间插入\verb|&|字符作为分割,在两行之间插入\verb|\\|作为分割。
\par
在上面一般形式的\verb|\begin{tabular}{c c}|里第二个大括号中的$c$是控制表中两列单元格里的内容居中显示的意思,另外还有$l$和$r$,前者控制单元格内容左对齐,后者控制内容右对齐。

\subsubsection{使用在线工具的注意事项}

提醒一下使用在线工具的读者,就是表格标题的问题,Tables Generator这一工具,它不仅不给浮动体控制,也并不曾添加表名及表名标签命令。\par 因此请一定注意给自己复制过来的代码里加上生成表名及表名标签的命令,也就是\verb|\chartname{表名}\label{表格标签}|,这样在后续论文的写作中才能引用它。

\subsubsection{表格范例}

下面做一些表格的示例,读者可参照本模板源码以及所编译生成的表格来进行简单的表格编写。

\subsubsection{普通表格}

下面是一些普通表格的示例:

\begin{table}[ht]
\begin{minipage}[t]{0.45\textwidth}
  \centering
  \caption{简单表格}
  \label{tab:1}
  \begin{tabular}{|l|c|r|}
    \hline
    姓甚名谁 & 何处高就 & 月入几何 \\
    \hline
    张三& 某腾 & 2W \\
    \hline
    李四& 某阿 & 1W \\
    \hline
    王五& 某团 & 3W \\
    \hline
  \end{tabular}
\end{minipage}
\begin{minipage}[t]{0.45\textwidth}
  \centering
  \caption{一般三线表}
  \label{tab:2}
  \begin{tabular}{ccc}
    \hline
    姓名& 学号& 性别\\
    \hline
    张三& 001& 男\\
    李四& 002& 女\\
    \hline
  \end{tabular}
\end{minipage}  
\end{table}

\begin{table}[ht]
\begin{minipage}[t]{0.45\textwidth}
  \centering
  \caption{无框线表格}
  \label{tab:3}
  \begin{tabular}{ c c c }
 cell1 & cell2 & cell3 \\ 
 cell4 & cell5 & cell6 \\  
 cell7 & cell8 & cell9    
\end{tabular}
\end{minipage}
\begin{minipage}[t]{0.45\textwidth}
\centering
\caption{组合行列表格}
\label{tab:4}
\begin{tabular}{ |c|c|c|c| } 
\hline
col1 & col2 & col3 \\
\hline
\multirow{3}{4em}{Multiple row} & cell2 & cell3 \\ 
& cell5 & cell6 \\ 
& cell8 & cell9 \\ 
\hline
\end{tabular}
\end{minipage}  
\end{table}

\begin{table}[!htb]
\centering
\caption{定长表}
\label{tab:5}
\begin{tabular}{ |p{3cm}||p{3cm}|p{3cm}|p{3cm}| }
 \hline
 \multicolumn{4}{|c|}{Country List} \\
 \hline
Country Name or Area Name& ISO ALPHA 2 Code & ISO ALPHA 3 Code & ISO numeric Code \\
 \hline
 Afghanistan   & AF    &AFG&   004\\
 Aland Islands&   AX  & ALA   &248\\
 Albania &AL & ALB&  008\\
 Algeria    &DZ & DZA&  012\\
 American Samoa&   AS  & ASM&016\\
 Andorra& AD  & AND   &020\\
 Angola& AO  & AGO&024\\
 \hline
\end{tabular}
\end{table}

更多的表格样式可以看看Overleaf的帮助文档(点击菜单拉到底),Overleaf的帮助文档是比较全面的,基本上数学公式的排版、图片插入、图像绘制、表格创建等论文写作可能会用到的内容它都有讲解,读者可以多在这一文档里看看。

\subsubsection{跨页表格}

跨页表格常用于附录(把正文放不下的实验数据统统放在附录的表中),以下是一个跨页表格的示例:

{\centering
  \begin{longtable}{ccccccccc}
  \caption{跨页表格示例} \\
  \toprule
  1     & 0 & 5  & 1  & 2  & 3  & 4  &  5 & 6 \\
  \midrule
  \endfirsthead

  \multicolumn{1}{l}{接上一页} \\
  \toprule
  1     & 0 & 5  & 1  & 2  & 3  & 4  &  5 & 6 \\
  \midrule
  \endhead

  \bottomrule
  \hline \\
  \multicolumn{9}{r}{{转下一页}} \\
  \endfoot

  \bottomrule
  \endlastfoot    

  1     & 0 & 5  & 1  & 2  & 3  & 4  &  5 & 6 \\
  1     & 0 & 5  & 1  & 2  & 3  & 4  &  5 & 6 \\
  1     & 0 & 5  & 1  & 2  & 3  & 4  &  5 & 6 \\
  1     & 0 & 5  & 1  & 2  & 3  & 4  &  5 & 6 \\
  1     & 0 & 5  & 1  & 2  & 3  & 4  &  5 & 6 \\
  1     & 0 & 5  & 1  & 2  & 3  & 4  &  5 & 6 \\
  1     & 0 & 5  & 1  & 2  & 3  & 4  &  5 & 6 \\
  1     & 0 & 5  & 1  & 2  & 3  & 4  &  5 & 6 \\
  1     & 0 & 5  & 1  & 2  & 3  & 4  &  5 & 6 \\
  1     & 0 & 5  & 1  & 2  & 3  & 4  &  5 & 6 \\
  1     & 0 & 5  & 1  & 2  & 3  & 4  &  5 & 6 \\
  1     & 0 & 5  & 1  & 2  & 3  & 4  &  5 & 6 \\
  1     & 0 & 5  & 1  & 2  & 3  & 4  &  5 & 6 \\
  1     & 0 & 5  & 1  & 2  & 3  & 4  &  5 & 6 \\
  1     & 0 & 5  & 1  & 2  & 3  & 4  &  5 & 6 \\
  1     & 0 & 5  & 1  & 2  & 3  & 4  &  5 & 6 \\
  1     & 0 & 5  & 1  & 2  & 3  & 4  &  5 & 6 \\
  1     & 0 & 5  & 1  & 2  & 3  & 4  &  5 & 6 \\
  1     & 0 & 5  & 1  & 2  & 3  & 4  &  5 & 6 \\
  1     & 0 & 5  & 1  & 2  & 3  & 4  &  5 & 6 \\
  1     & 0 & 5  & 1  & 2  & 3  & 4  &  5 & 6 \\
  1     & 0 & 5  & 1  & 2  & 3  & 4  &  5 & 6 \\
  1     & 0 & 5  & 1  & 2  & 3  & 4  &  5 & 6 \\
  1     & 0 & 5  & 1  & 2  & 3  & 4  &  5 & 6 \\
  1     & 0 & 5  & 1  & 2  & 3  & 4  &  5 & 6 \\
  1     & 0 & 5  & 1  & 2  & 3  & 4  &  5 & 6 \\
  1     & 0 & 5  & 1  & 2  & 3  & 4  &  5 & 6 \\
  1     & 0 & 5  & 1  & 2  & 3  & 4  &  5 & 6 \\

  \end{longtable}
}

\subsection{列表}

\subsubsection{列表环境}

\begin{tabular}{l l}
\toprule
环境命令 & 命令含义\\
\midrule
\verb|\begin{enumerate}...\end{enumerate}| & 有序列表 \\
\verb|\begin{itemize}...\end{itemize}| & 无序列表 \\
\verb|\begin{description}...\end{description}| & 描述性列表 \\
\bottomrule
\end{tabular}

在一个列表环境中可以继续嵌套列表环境以实现更丰富的表现形式。

在列表环境中每个条目均以\verb|\item|命令开头,一个\verb|\item|即为一个条目。

当需要更改条目前的序号或符号时可以使用\verb|\item[...]|命令,在中括号中输入想要的序号或者符号即可。这一命令在描述性列表环境中,输入的应为描述性的文本内容。

\subsubsection{列表范例}

下面演示了创建有序列表、无序列表以及描述性列表,读者可参见本模板源码进行简单的列表编写。更多示例参见\href{https://www.latex-tutorial.com/tutorials/lists/}{https://www.latex-tutorial.com/tutorials/lists/}{LaTeX Lists}。

\subsubsection{有序列表}

有序列表即一个计数的使用序号来标识各条目的列表。
  \begin{enumerate}
      \item 第一条目
          \begin{enumerate}
              \item 第一条目中的第一项
              \item 第一条目中的第二项
          \end{enumerate}
      \item 第二条目
    \begin{enumerate}
      \item 第二条目中的第一项
      \item 第二条目中的第二项
    \end{enumerate}
      \item 第三条目
  \end{enumerate}

\subsubsection{无序列表}
  无序列表即一个不计数而使用符号来标识各条目的列表。
  \begin{itemize}
      \item 第一条目
      \begin{itemize}
          \item 第一条目中的第一项
          \item 第一条目中的第二项
      \end{itemize}
      \item 第二条目
      \item 第三条目
  \end{itemize}

\subsubsection{描述性列表}

描述性列表即一个用描述性文本来标识各条目的列表。

\begin{description}
  \item[Step 1]  条目一.
  \item[第二条]  条目二.
  \item[第三点]  条目三.
  \item[fourth]  条目四.
  \item[条目五]  条目五.
\end{description}

\subsection{参考文献}

\subsubsection{将文献添加到模板}

文献引用需要配合BibTeX使用,很多工具可以直接生成BibTeX文件(EndNote, NoteExpress, \href{https://xueshu.baidu.com/}{https://xueshu.baidu.com/}{百度学术},\href{https://ac.scmor.com/}{https://ac.scmor.com/}{谷歌学术})。\par 以百度学术为例,在其中找到一篇论文,点击\textbf{<引用>}按钮。在弹出来的窗口中点击导出至\textbf{<BibTeX>}按钮,将跳转页面的全部内容复制。 打开本模板ref.bib文件,将复制的内容空一行粘贴进去即可。

读者可以在每一篇文献的BibTeX内容代码的第一行大括号后进行命名,作用等同于引用其他内容所使用的标签。

\subsubsection{参考文献的罗列顺序}

本模板对参考文献的排序采用顺序编码制,也即文末参考文献的罗列顺序是与论文中的引用次序有关的,读者在论文中引用的第一篇文献会在参考文献页排在第一条,引用的第二篇文献则居于第二条,以此类推。

\subsection{论文中的引用}

\subsubsection{添加标签}

所谓标签也即是一种标识,为~\LaTeX~指明所需引用的对象。

故若要引用某一内容,需在该内容后添加标签,添加标签的命令为\verb|\label{...}|,大括号内为自行设定的标签文本。

添加标签时应注意标签的位置,例如一个编号的公式环境中只有一个公式时,标签命令置于公式内容之后或置于\verb|\begin{...}|之后,效用相同。但若是公式环境中有多个编号公式,则应将标签命令置于具体想要引用的那个公式之后。

对于图表的标签,一般均置于图名及表名命令之后。

\subsubsection{引用命令}

\begin{tabular}{l l}
  \verb|\ref{...}|   & 不带括号的引用 \\
  \verb|\eqref{...}|  & 引用公式所用的带括号引用 \\
  \verb|\pageref{...}| & 引用对象所在页码 \\
  \verb|\cite{...}| & 引用参考文献 
\end{tabular}

\subsubsection{插图的引用}

引用命令为\verb|\ref{...}|。

就像这样,如图\ref{fig:1},其描绘的是一羽扇纶巾的人。\par 

\subsubsection{表格的引用}

引用命令为\verb|\ref{...}|。

就像这样:如表~\ref{tab:6}所示,统计表格一般是三线表形式。

\subsubsection{页码的引用}

引用命令为\verb|\pageref{...}|。

就像这样,就如上文第\pageref{fig:1}页的图\ref{fig:1}所示,羽扇是黑色的。

\subsubsection{文献的引用}

引用命令为\verb|\cite{...}|。

本模版引用参考文献统一采用数字上标的形式\cite{r1}。

引用多个文献时只需将各文献的标签以逗号分隔\cite{r1,r2,r3,r4,r5}。

特别说明,这里的文献引用,是从本模板的文献数据库,也即是$ref.bib$中引用,而非从网站或学术网站引用,读者需要将自己撰写论文所需的参考文献的BibTeX信息导入$ref.bib$文件中方可实现在论文中引用。 % 导入第二章内容

\section{\LaTeX 进阶使用}
~\LaTeX~的简单使用阅读完第二章的内容已经足够,以下列举一些对学习~\LaTeX~进阶使用或者对更好的使用~\LaTeX~会有帮助的链接。
\begin{itemize}
\item \href{https://liam.page/2014/09/08/latex-introduction/}{https://liam.page/2014/09/08/latex-introduction/}{一份其实很短的~\LaTeX~入门文档}

\item \href{https://www.latexstudio.net/archives/51802.html}{https://www.latexstudio.net/archives/51802.html}{~\LaTeX~工作室入门学习文档}

\item \href{https://www.cnblogs.com/Coolxxx/p/5982439.html}{https://www.cnblogs.com/Coolxxx/p/5982439.html}{~\LaTeX~符号命令大全}

\item \href{https://cn.overleaf.com/learn}{https://cn.overleaf.com/learn}{Overleaf帮助文档}

\item \href{http://detexify.kirelabs.org/classify.html}{http://detexify.kirelabs.org/classify.html}{手写符号识别}

\end{itemize}
 % 导入第三章内容

\section{ 选题背景与研究意义}
\subsection{研究背景}
高光谱遥感(Hyperspectral Remote Sensing,HSI)全称为高光谱分辨率遥感,起源于20世纪80年代。近年来,随着航空航天技术的飞速发展,遥感系统采集到的影像光谱分辨率得到极大提升,光谱分辨率可以达到纳米级,航天成像光谱仪能够在连续的几十个甚至几百个光谱波段获取地物辐射信息。在取得地物空间图像同时,每个像元都能得到一条包含地物光谱特征的连续光谱曲线。

传统自然图像识别主要是通过图像中的纹理信息,在HSI中不同波段对同一地物会有不同的反射值,不同地物的光谱信息不同(如图)。高光谱遥感技术通过将地物光谱信息和纹理信息结合,从而实现对地物的精确检测。HSI紧密结合了遥感图像和光谱信息,成为人类认知和感知世界一种新的手段。

\subsection{ 研究意义}

在信息化时代的今天,高光谱遥感技术已经和人类工作生活息息相关。高光谱遥感突出特点和优势使其在众多领域发挥着越来越重要的作用,比如在军事目标侦查、阵地与装备伪装识别等\cite{yang2017};在农业研究方面,通过高光谱数据能够反演作物物理和化学参数,进行作物长势监测、品质评估等\cite{liu2020};在环境监测方面,将可见光、近红外、热红外和高光谱技术与实际测得的地面光谱数据相结合,以分析土壤属性及其与土壤光谱反射率的定量关系\cite{tian2019}。
随着高光谱技术进入人们生活的方方面面,这些广泛的应用要求HSI的分类结果更加准确。其中,高光谱图像特征提取与识别分类是解译和分析HSI的关键技术,因此,关于HSI的特征提取与分类识别不仅具有重大学术研究价值还具备重要现实意义。 % 导入第四章内容

%%%===============================加入结论=====================================%%%
\section{总结与展望}     
%在下面进行结论的撰写,除了本行的命令,下面的内容可随意更改
结论是一篇学位论文的收尾部分,是以研究成果为前提,经过严密的逻辑推理和论证所得出最终的、总体的结论。换句话说,结论应是整篇论文的结局,而不是某一局部问题或某一分支问题的结论。结论应体现学生更深层的认识,且从全篇论文的全部材料出发,经过推理、判断、归纳等逻辑分析过程而得到的新的学术总观念、总见解。

结论是论文主要成果的总结,客观反映了论文或研究成果的价值。论文结论与问题相呼应,同摘要一样可为读者和二次文献作者提供依据。结论的内容不是对研究结果的简单重复,而是对研究结果更深人一步的认识‘是从正文部分的全部内容出发,并涉及引言的部分内容,经过判断、归纳、推理等过程而得到的新的总观点。毕业论文的研究结论通常由三部分构成:研究结论、不足之处、后续研究或建议。

第一,毕业论文的结论主要是由研究的背景与问题、文献综述、研究方法、案例资料分析与整理等研究得到的,其中核心的结论是正文部分的资料分析与研究的结果得出的结论和观点,即论文的基本结论。本研究结论说明了什么问题,得出了什么规律性的东西,解决了什么实际问题。研究结论必须淸楚地表明本论文的观点,有什么理论背景的支持,对实践有什么指导意义等,若用数字来说明则效果嫌佳,说服力最强。不能模棱两可,含糊其辞。避免使人有似是而非的感觉,从而怀疑论文的真正价值。 

第二,研究的不足,表明本论文的局限性所在,包括研究假设、资料收集、研究方法方面的不足之处,可以为后来的研究在该领域进一步完善指明方向。
对于一篇学位论文的结论,上述基本结论是必需的,而不足之处和研究建议则视论文的具体内容可以多论述或少论述。论文的结论部分具有相对的独立性,应提供明确、具体的定性和定量信息。可读性要强。

%%%=============================加入参考文献====================================%%%
\clearpage
\fancyhead[C]{\song \zihao{5} 参考文献} % 设置页眉
\renewcommand{\headrulewidth}{0.5pt} %设置页眉横线粗细
\phantomsection\addcontentsline{toc}{section}{\heiti \zihao{-4} \textbf{参考文献}} % 将参考文献添加到目录
\bibliographystyle{gbt7714-numerical}
\bibliography{references}
%%%%===============================加入致谢=====================================%%%
%%% --------------- 致谢 ------------- - %%%
\fancyhead[C]{致\quad 谢} % 设置页眉
\renewcommand{\headrulewidth}{0.5pt} %设置页眉横线粗细
\addcontentsline{toc}{section}{\heiti \zihao{-4} \textbf{致\quad 谢} }\tolerance=500 %将摘要放进目录
\section*{\heiti \zihao{3} \textbf{致\quad 谢}}
\kai \zihao{-4}
人这辈子会有很多需要感谢人、事、物。

大学这些里,帮忙办手续的老师、送你去医院的室友、答疑解惑的导师、春天温润的雨、与朋友们吃喝谈笑的那个夏日傍晚的风……总有让你难以忘怀的,铭刻心间的。

不妨在这里暂且停下,不要想那些琐碎与焦躁,细细想想,每一个曾给予帮助的人,每一件让自己有所成长的事物,不必纠结辞藻,只需直抒胸臆,只需将真心实意小小流露。
















       
%%%%=============================加入研究成果====================================%%%
\fancyhead[C]{\song \zihao{5} 攻读学位期间的研究成果} % 设置页眉
\renewcommand{\headrulewidth}{0.5pt} %设置页眉横线粗细
\addcontentsline{toc}{section}{\heiti \zihao{-4} \textbf{攻读学位期间的研究成果}}\tolerance=500 %将摘要放进目录
\section*{\heiti \zihao{3} \textbf{攻读学位期间的研究成果}}
\songti \zihao{-4} \textbf{已发表论文}:
\par 
\begin{enumerate}
\songti \zihao{5}
\item XXX,XX. 国内经济相互作用研究[J]. 有色金属科学与工程,2010,21(3): 70-74.

\item XXX,XX. 风化壳淋积型稀土矿提取除杂技术现状及进展[J]. 稀土,2011 (已录用).

\end{enumerate}

       

\end{document}
