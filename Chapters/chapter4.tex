\section{ 选题背景与研究意义}
\subsection{研究背景}
高光谱遥感(Hyperspectral Remote Sensing,HSI)全称为高光谱分辨率遥感,起源于20世纪80年代。近年来,随着航空航天技术的飞速发展,遥感系统采集到的影像光谱分辨率得到极大提升,光谱分辨率可以达到纳米级,航天成像光谱仪能够在连续的几十个甚至几百个光谱波段获取地物辐射信息。在取得地物空间图像同时,每个像元都能得到一条包含地物光谱特征的连续光谱曲线。

传统自然图像识别主要是通过图像中的纹理信息,在HSI中不同波段对同一地物会有不同的反射值,不同地物的光谱信息不同(如图)。高光谱遥感技术通过将地物光谱信息和纹理信息结合,从而实现对地物的精确检测。HSI紧密结合了遥感图像和光谱信息,成为人类认知和感知世界一种新的手段。

\subsection{ 研究意义}

在信息化时代的今天,高光谱遥感技术已经和人类工作生活息息相关。高光谱遥感突出特点和优势使其在众多领域发挥着越来越重要的作用,比如在军事目标侦查、阵地与装备伪装识别等\cite{yang2017};在农业研究方面,通过高光谱数据能够反演作物物理和化学参数,进行作物长势监测、品质评估等\cite{liu2020};在环境监测方面,将可见光、近红外、热红外和高光谱技术与实际测得的地面光谱数据相结合,以分析土壤属性及其与土壤光谱反射率的定量关系\cite{tian2019}。
随着高光谱技术进入人们生活的方方面面,这些广泛的应用要求HSI的分类结果更加准确。其中,高光谱图像特征提取与识别分类是解译和分析HSI的关键技术,因此,关于HSI的特征提取与分类识别不仅具有重大学术研究价值还具备重要现实意义。